\documentclass[a4paper]{article}

%% Language and font encodings
\usepackage[english]{babel}
\usepackage[utf8x]{inputenc}
\usepackage[T1]{fontenc}

%% Sets page size and margins
\usepackage[a4paper,top=3cm,bottom=2cm,left=3cm,right=3cm,marginparwidth=1.75cm]{geometry}

%% Useful packages
\usepackage{amsmath}
\usepackage{graphicx}
\usepackage[colorinlistoftodos]{todonotes}
\usepackage[colorlinks=true, allcolors=blue]{hyperref}

\title{A Stochastic Model for Origin-Destination Estimation}
\author{Hao Hao}

\begin{document}
\maketitle

\begin{abstract}
Your abstract.
\end{abstract}

\section{Introduction}

The future works are as following:
\begin{enumerate}
  \item Implementation Focused(Python, Matlab, Julia)
  \item Deterministic Model or Stochastic Model
  \item Optimisation: eliminate model constrains
  \item Make the convergence faster  
\end{enumerate}

The purpose of a research placement
The aim of a research placement is different from that of an industrial placement, especially when the placement takes place within academia.
Whilst in an industrial placement, the employer sets the task and the employee completes the task in a professional way based on his/her available knowledge or knowledge gathered from other sources. A research placement on the other hand focuses much more on the formulation and testing of new ideas and hypotheses and the discovery of novel approaches. In these cases there is not yet a solution to the problem available and a clear methodology of finding solutions needs to be formulated. Research in academia distinguishes itself from research in industry in that the researcher needs to define his/her own research questions. In addition, research in academia focuses very much on public dissemination via conference and journal publications whilst that in industry focuses much more on patent generation. Academic careers also need to consider the aspect of teaching future generations of engineers for both industrial as well as academic settings.
A research placement is very much about discovery and the identification of ones own role within the discovery process.
As a consequence of the distinction between an industrial placement and a research placement, the evaluation criteria are also different as the learning outcomes are different.

\vspace{5mm} 

The learning outcomes of a research placement are, learning how to:

1. formulate a new idea/hypothesis/problem and evaluate whether the question posed solves a problem that has not yet been solved before and requires solving.

2. build/design a methodology that will lead to the finding of a definitive and comprehensive answer that can be benchmarked, proven and checked.
3. find appropriate tools and performance parameters to analyze and deliver justifiable results.

4. discover and formulate ones own contributions and innovations within a team work setting.

5. disseminate findings to the wider public via in-house, national or international conference and/or journal publications.

\vspace{5mm} 
Important topics for interim report:

* Define the research topic: define specific research questions/problems/hypotheses

* Give the background literature that supports the research and identifies the gaps in the knowledge that defines the novelty of the research

* Develop the research methods. Design the appropriate methodology (simulations, experiments, benchmark parameters, …) that will allow the gathering of evidence and the formulation of the conclusion.

* Give the criteria against which to evaluate progress and success.
Important topics for final report:

* Write a conference of journal paper that could be submitted for publication in the scientific community.

* Identify the reasons why some research methodologies applied where appropriate/inappropriate for the research tasks and give an improvement/alternative of a more effective methodology to achieve the research outcomes.

* Reflect on your own contribution: where did you add to the group effort (e.g. because you measured/simulated a certain set of items the group was able to publish, move forward faster…), where did you propose and test alternative approaches and which of those where implemented or not (why),

* What did you learn? Describe how you see research: aims, tasks and outcomes.

\section{Literature Review}


\begin{center}
\begin{tabular}{ | p{7cm} | p{7cm}|  } 
\hline
Literature & Method \\
\hline
Estimation of an origin-destination matrix with random link choice proportions: A statistical approach(Lo,Zhang,Lam 1995) & 
Simultaneous estimation of OD-flow matrix and assignment matrix: 1)MLE estimator:Assume OD flows and link choice proportions are deterministic; 2)Bayesian approach:Assume OD flows and link choice proportions are random variables whose particular realisations we must estimate   \\ 
\hline
Simultaneous estimation of an origin-destination matrix and link choice proportions using traffic counts(Lo,Chan 2003) 
& 
Introduce algorithm which obtains estimates utilising the previous established MLE method, the new estimators obtained during each iteration is used to establish the likelihood function during the next iteration. 1)Introduce the calculation of the link choice proportions using weight, link capacity and dispersion parameter; 2) Iterative algorithm for convergence \\ 
\hline
cell7 & cell8  \\ 
\hline
\end{tabular}
\end{center}






\end{document}